%%%%%%%%%%%%%%%%%%%%%%%%%%%%%%%%%%%%%%%%%%%%%%%%%%%%%%%%%%%%%%%%%%%%%%%%
%%%%%%%%%%%%%%%%%%%%%%%%%%% Hugos Template %%%%%%%%%%%%%%%%%%%%%%%%%%%%%
%%%%%%%%%%%%%%%%%%%%%%%%%%%%%%%%%%%%%%%%%%%%%%%%%%%%%%%%%%%%%%%%%%%%%%%%
%%%   Settings

%   Ändra språk och snabb compile
\usepackage[swedish]{babel}       % Språk. [swedish] eller [english].
\usepackage[]{graphicx}           % Bilder. [demo] för snabb compile!

%   Extra
%\usepackage{mhchem}              % Kemi Ekvationer.
%\usepackage{datetime}            % Flera datum.



%%%%%%%%%%%%%%%%%%%%%%%%%%%%%%%%%%%%%%%%%%%%%%%%%%%%%%%%%%%%%%%%%%%%%%%%
%%%   Packages

\usepackage[utf8]{inputenc}  % Kodning av text.
\usepackage[]{biblatex}      % Referenser. [style=apa]
\pagenumbering{gobble}       % Stoppar sidnumrering på titelsida.

\usepackage{csquotes}        % Quotes.
\usepackage{mathtools}       % Ekvationer.
\usepackage[]{geometry}      % Sidlayout.
\usepackage{float}           % För [H].
\usepackage[T1]{fontenc}     % Bra för åäö. Kopiering.


\bibliography{Essential/References.bib} % Mapp för Källor.
\graphicspath{{Images/}}                % Mapp för bilder.
\usepackage[none]{hyphenat}             % Slutar dela upp ord.
\usepackage{pgffor}                     % Loop för bilagor.
\usepackage{pgfmath}                    % Används i bilagaloop.
\setlength{\parindent}{0cm}             % Tar bort indent på ny rad.
\renewcommand{\arraystretch}{1.2}       % Gör tabeller större.
\usepackage{pdfpages}                   % Pdfs.
\usepackage[font={small,it},labelfont=bf%
,justification=centering]{caption}      % Snygg figurtext.
\setlength{\abovecaptionskip}{3pt}      % Flyttar tabelltext närmare.
\usepackage[per-mode=symbol]{siunitx}   % si. Skriva enheter/nummer.
\sisetup{output-decimal-marker={,},%    % si. (Ändra "{,}" till "{.}" för engelska).
range-phrase=--,range-units=single,exponent-product=\cdot}

%   Hyperlänkar + färg
\usepackage[colorlinks]{hyperref}
\usepackage[nameinlink,noabbrev,swedish]{cleveref} % \cref{}
\usepackage{color}
\definecolor{royalblue}{rgb}{0.0, 0.14, 0.4}
\hypersetup{
     colorlinks  = true,
     linkcolor   = royalblue,          % internal links
     citecolor   = royalblue,          % bibliography
     filecolor   = royalblue,          % file links
     urlcolor    = royalblue           % external links
}

%%%%%%%%%%%%%%%%%%%%%%%%%%%%%%%%%%%%%%%%%%%%%%%%%%%%%%%%%%%%%%%%%%%%%%%%
%%%   Bilaga loop

% Kollar om det finns .tex fil annars .pdf. Från Bilaga_1 till Bilaga_50.
% Genererar sida om det finns.
\newcommand{\bilagaloop}{\foreach \i in {1, 2, 3, ...,50} {%
    \edef\FileName{Sections/Bilagor/Bilaga_\i}%
    \IfFileExists{\FileName.tex}{%
% Skapa .tex sida
    \newpage%
    \setcounter{page}{1}%
    \section*{Bilaga \i, \pgfmathparse{\bilaganamn[\i-1]}\pgfmathresult}%
    \input{\FileName.tex}% 
%
       }{\IfFileExists{\FileName.pdf}{%
% Skapa .pdf sida
       \newpage%
       \setcounter{page}{1}%
       \includepdf[pagecommand=\section*{Bilaga \i,  \pgfmathparse{\bilaganamn[\i-1]}\pgfmathresult} ,scale=0.9,offset=0mm -35]{\FileName.pdf}%
%
       }{}}}}


%%%%%%%%%%%%%%%%%%%%%%%%%%%%%%%%%%%%%%%%%%%%%%%%%%%%%%%%%%%%%%%%%%%%%%%%
%%%%%%%%%%%%% Creative Commons CC BY 4.0, Hugo Laestander %%%%%%%%%%%%%%
%%%%%%%%%%%%%%%%%%%%%%%%%%%%%%%%%%%%%%%%%%%%%%%%%%%%%%%%%%%%%%%%%%%%%%%%