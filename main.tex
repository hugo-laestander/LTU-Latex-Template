%%%%%%%%%%%%%%%%%%%%%%%%%%%%%%%%%%%%%%%%%%%
%%%   Packages
\documentclass[12pt]{article}
%%%%%%%%%%%%%%%%%%%%%%%%%%%%%%%%%%%%%%%%%%%%%%%%%%%%%%%%%%%%%%%%%%%%%%%%
%%%%%%%%%%%%%%%%%%%%%%%%%%% Hugos Template %%%%%%%%%%%%%%%%%%%%%%%%%%%%%
%%%%%%%%%%%%%%%%%%%%%%%%%%%%%%%%%%%%%%%%%%%%%%%%%%%%%%%%%%%%%%%%%%%%%%%%
%%%   Settings

%   Ändra språk och snabb compile
\documentclass[12pt]{article}     % Storlek på text i dokumentet.
\usepackage[swedish]{babel}       % Språk. [swedish] eller [english]
\usepackage[]{graphicx}           % Bilder. [demo] för snabb compile!

%   Namnge bilagor
\edef\bilaganamn{{"Intressant data","namn 2 osv"}}

%   Extra
%\usepackage{mhchem}              % Kemi Ekvationer.
%\usepackage{datetime}            % Flera datum.



%%%%%%%%%%%%%%%%%%%%%%%%%%%%%%%%%%%%%%%%%%%%%%%%%%%%%%%%%%%%%%%%%%%%%%%%
%%%   Packages

\usepackage[utf8]{inputenc}  % Kodning av text.
\usepackage[]{biblatex}      % Referenser.
\pagenumbering{gobble}       % Stoppar sidnumrering på titelsida.


\usepackage{csquotes}        % Quotes.
\usepackage{mathtools}       % Ekvationer.
\usepackage[]{geometry}      % Sidlayout.
\usepackage{float}           % för [H].

\bibliography{Essential/references.bib} % Källor.
\graphicspath{{Images/}}                % Mapp för bilder.
\usepackage[none]{hyphenat}             % Sluta dela upp ord.
\usepackage{pgffor}                     % Loop för bilagor
\setlength{\parindent}{0cm}             % Tar bort indent på ny rad.
\renewcommand{\arraystretch}{1.2}       % Gör tabeller större.
\usepackage{pdfpages}                   % Pdfs.
\usepackage{pgfmath}                    % Används för bilaga loop
\usepackage[font={small,it},labelfont=bf%
,justification=centering]{caption}      % Snygg figurtext

%   Hyperlänkar + färg
\usepackage[colorlinks]{hyperref}            
\usepackage{color}
\definecolor{royalblue}{rgb}{0.0, 0.14, 0.4}
\usepackage[nameinlink,noabbrev,swedish]{cleveref} % Ref ekv, fig etc
\hypersetup{
     colorlinks  = true,
     linkcolor   = royalblue,          % internal links
     citecolor   = royalblue,          % bibliography
     filecolor   = royalblue,          % file links
     urlcolor    = royalblue           % external links
}

%%%%%%%%%%%%%%%%%%%%%%%%%%%%%%%%%%%%%%%%%%%%%%%%%%%%%%%%%%%%%%%%%%%%%%%%
%%%   Bilaga loop

% Kollar om det finns .tex fil annars .pdf. Från Bilaga_1 till Bilaga_99.
% Genererar sida om det finns.
\newcommand{\bilagaloop}{\foreach \i in {1, 2, 3, ...,99} {%
    \edef\FileName{Sections/Bilagor/Bilaga_\i}%
    \IfFileExists{\FileName.tex}{%
% Skapa .tex sida
    \newpage%
    \setcounter{page}{1}%
    \section*{Bilaga \i, \pgfmathparse{\bilaganamn[\i-1]}\pgfmathresult}%
    \input{\FileName.tex}% 
%
       }{\IfFileExists{\FileName.pdf}{%
% Skapa .pdf sida
       \newpage%
       \setcounter{page}{1}%
       \includepdf[pagecommand=\section*{Bilaga \i,  \pgfmathparse{\bilaganamn[\i-1]}\pgfmathresult} ,scale=0.9,offset=0mm -35]{\FileName.pdf}%
%
       }{}}}}


%%%%%%%%%%%%%%%%%%%%%%%%%%%%%%%%%%%%%%%%%%%%%%%%%%%%%%%%%%%%%%%%%%%%%%%%
%%%%%%%%%%%%% Creative Commons CC BY 4.0, Hugo Laestander %%%%%%%%%%%%%%
%%%%%%%%%%%%%%%%%%%%%%%%%%%%%%%%%%%%%%%%%%%%%%%%%%%%%%%%%%%%%%%%%%%%%%%%
%   Namnge bilagor
\edef\bilaganamn{{"Intressant data","namn 2 osv",""}}

%%%   Title
\begin{document}
\newgeometry{top=3cm, hmargin=0cm, bottom=1.5cm}
%%%%%%%%%%%%%%%%%%%%%%%%%%%%%%%%%%%%%%%%%%%%%%%%%%%%%%%%%%%%%
%   Skapar variabler, ändra dessa!

% Titel
\def\thetitle{En Bra Titel}
\def\theundertitle{Beskrivande Undertitel}

% Bild
\def\thefrontpage{framsida.png}

% Kurs
\def\thecourse{Kursnamn, F0000T}

% Författare
\def\theauthor{
Namn (abcdef-7@student.ltu.se) \\
Namn (abcdef-7@student.ltu.se) \\ 
Namn (abcdef-7@student.ltu.se)
}

% Handledare
\def\thesupervisor{Albert Einstein}

% Institution
\def\theinstitution{Institutionen för teknikvetenskap och matematik}

%%%%%%%%%%%%%%%%%%%%%%%%%%%%%%%%%%%%%%%%%%%%%%%%%%%%%%%%%%%%%
%   Detta genererar titelsidan

\begin{titlepage}
	\centering
	
	% Titel
	{\Huge \textrm{\thetitle} \par}
	{\Large \textrm{\theundertitle} \par}
	\vspace{0.8cm}
	
	% Bild, ändra "height" om bilden för stor/liten
	\includegraphics[height=0.45\textwidth]{\thefrontpage}\par
	\texttt{\thecourse}\par
	\vspace{0.8cm}
	
	% Namn
	{\textbf{Författare:} \par}
	{\large \theauthor\par} % Ta bort \large ifall de behövs
	\vspace{0.8cm}
	\textbf{Handledare:}\par
	\large{\thesupervisor}  % Ta bort \large ifall de behövs
	\vfill
	
	% logga, datum
    \includegraphics[width=0.2\textwidth]{Images/ltu_swe.jpg} \\
    \vspace{0.2cm}
    \textrm{\theinstitution} \\
	{\large \textrm{\today}\par}
\end{titlepage}
\restoregeometry
\pagenumbering{roman}
\setcounter{page}{2}


%%%%%%%%%%%%%%%%%%%%%%%%%%%%%%%%%%%%%%%%%%%
%%%   Sections
% Radera filer i "Sections/" som ej behövs. 
% Då försvinner de automatiskt ur dokumentet.
% Vid behov, ändra titlar/filer osv nedanför.

\IfFileExists{Sections/Sammanfattning.tex}{
\section*{Sammanfattning}
Sammanfattning av rapporten på 150-200 ord
}{}

\IfFileExists{Sections/Abstract.tex}{
\section*{Abstract}
\section*{Abstract}
%
Sammanfattning på engelska.
}{}

\newpage
\tableofcontents
\newpage

% table of contents för figurer och tabeller
%\listoffigures
%\listoftables
%\newpage

\IfFileExists{Sections/Beteckningar.tex}{
\section*{Beteckningar}
\begin{raggedleft}
 \begin{tabular}{ c  l  c }
 \textbf{Symbol} & \textbf{Beskrivning} & \textbf{Enhet} \\ [0.5ex]
 \hline
 $E$ & Energi & \si{J} \\
 \hline
   $m$ & Massa & \si{kg} \\
 \hline
  $c$ & Ljusets hastighet & \si{m/s} \\
 \hline
\end{tabular}
\end{raggedleft}
}{}

\newpage
\pagenumbering{arabic}

\IfFileExists{Sections/Inledning.tex}{
\newpage
\section{Inledning}
\section{Inledning} \label{s:inledning}
%
Det här är inledning.
}{}

\IfFileExists{Sections/Teori.tex}{
%\newpage
\section{Teori}
\section{Teori} \label{s:teori}
%
Det här är\footfullcite{einstein} teori \cite{einstein}. \n

Hänvisning till \cref{eq:namn}
%
\begin{equation} \label{eq:namn}
    E=m \cdot c^2.
\end{equation}
%
Hänvisning till \cref{tab:namn}.
%
\begin{table}[H]
\centering
\caption{Tabellhuvud.}
\begin{tabular}{@{} l l l @{}} \toprule
\textbf{Symbol} & \textbf{Storhet} & \textbf{Dimension} \\
\midrule
    $E$ & Energi & \si{M.L^2.T^{2}} \\
    $m$ & Massa &\si{M} \\
    $c$ & Ljusets hastighet & \si{L.T^{-1}} \\
\bottomrule 
\end{tabular} \label{tab:namn}
\end{table}
%


}{}

\IfFileExists{Sections/Metod.tex}{
%\newpage
\section{Metod}
\section{Metod}
Det här är metod.
}{}

\IfFileExists{Sections/Resultat.tex}{
%\newpage
\section{Resultat}
\section{Resultat} \label{s:resultat}
%
Det här är resultat.
}{}

\IfFileExists{Sections/Diskussion_s.tex}{
%\newpage
\section{Diskussion och slutsatser}
\section{Diskussion och slutsatser}
Det här är diskussion och slutsatser.
}{}

%\newpage
\printbibliography

%%%%%%%%%%%%%%%%%%%%%%%%%%%%%%%%%%%%%%%%%%%
%%%   Bilagor
% Skapa nya filer i "Sections/Bilagor/"
% som heter Bilaga_1, Bilaga_2, ... Så läggs de in automatiskt!
% Funkar för .tex och .pdf.
% Namnge bilagor högst upp ^.
\IfFileExists{Sections/Bilagor/Bilaga_1.tex}{\bilagaloop}
{\IfFileExists{Sections/Bilagor/Bilaga_1.pdf}{\bilagaloop}{}}
% Koden kollar ifall den första filen finns, annars körs den ej :)

\end{document}