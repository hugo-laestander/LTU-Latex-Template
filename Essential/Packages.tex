%-----------------------------------------------
%%              Hugos Template 1.13
%-----------------------------------------------
%%                  Settings
%-----------------------------------------------
%   Ändra språk och snabb compile
\usepackage[swedish]{babel}       % Språk. [swedish] eller [english].
\usepackage[]{graphicx}           % Bilder. [demo] för snabb compile!

%   Extra
%\usepackage{mhchem}              % Kemi Ekvationer.


%-----------------------------------------------
%%                  Packages
%-----------------------------------------------
\usepackage[utf8]{inputenc}  % Kodning av text.
\usepackage[]{biblatex}      % Referenser. [style=apa]
\pagenumbering{gobble}       % Stoppar sidnumrering på titelsida.

\usepackage{csquotes}        % Quotes.
\usepackage{mathtools}       % Ekvationer.
\usepackage[]{geometry}      % Sidlayout.
\usepackage{float}           % För [H].
\usepackage[T1]{fontenc}     % Bra för åäö. Kopiering.
\usepackage{microtype}       % Fixar bättre textlayout.

\usepackage[font={small,it},labelfont=bf%
,justification=centering]{caption}      % Snygg figurtext.
\usepackage{booktabs}                   % Snygga Tabeller
\setlength{\abovecaptionskip}{3pt}      % Flyttar tabelltext närmare.
\renewcommand{\arraystretch}{1.2}       % Gör tabeller större.
\setlength{\parindent}{0cm}             % Tar bort indent på ny rad.
\usepackage[none]{hyphenat}             % Slutar dela upp ord.

\bibliography{Essential/References.bib} % Mapp för Källor.
\graphicspath{{Images/}}                % Mapp för bilder.
\usepackage{pgffor}                     % Loop för bilagor.
\usepackage{pdfpages}                   % Pdfs.
\addto{\captionsswedish}{\renewcommand*%
{\contentsname}{Innehållsförteckning}}  % Ändrar namn -> Innehållsförteckning

\usepackage[colorlinks]{hyperref}                  % Hyperlänkar + färg
\usepackage[nameinlink,noabbrev,swedish]{cleveref} % \cref{} & \Cref{}
\usepackage{color}                                 % Färg
\definecolor{royalblue}{rgb}{0.0, 0.14, 0.4}       % Egen färg
\hypersetup{                                       % Färg på hyperlänkar
     colorlinks  = true,
     linkcolor   = black,              % internal links
     citecolor   = royalblue,          % bibliography
     filecolor   = royalblue,          % file links
     urlcolor    = royalblue           % external links
}

\usepackage{matlab-prettifier}          % Matlab text!
\usepackage[per-mode=symbol]{siunitx}   % si. Skriva enheter/nummer.
\sisetup{output-decimal-marker={,},%    % si. ("{,}" till "{.}" för engelska).
range-phrase=--,range-units=single,exponent-product=\cdot}

%-----------------------------------------------
%%                New commands
%-----------------------------------------------
\newcommand{\n}{\vskip 1em}

%-----------------------------------------------
%%                Bilaga loop
%-----------------------------------------------
% Kollar om det finns .tex fil Från Bilaga_1 till Bilaga_25.
% Genererar sida om det finns.
\newcommand{\bilagaloop}{\foreach \i in {1, 2, 3, ...,25} {%
    \edef\FileName{Sections/Bilagor/Bilaga_\i}%
    \IfFileExists{\FileName.tex}{%
% Skapa .tex sida
    \newpage%
    \setcounter{page}{1}%
    \input{\FileName.tex}%
}{}}}

%%%%%%%%%%%%%%%%%%%%%%%%%%%%%%%%%%%%%%%%%%%%%%%%%%%%
%%% Creative Commons CC BY 4.0, Hugo Laestander %%%%
%%%%%%%%%%%%%%%%%%%%%%%%%%%%%%%%%%%%%%%%%%%%%%%%%%%%